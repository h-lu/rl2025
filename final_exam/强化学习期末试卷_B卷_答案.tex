\documentclass[zihao=5,answers]{BHCexam}
\usepackage{ctex}
\usepackage{multirow}

%\linespread{1.2}

% \usepackage{minted}
% \setminted{mathescape, style = black, linenos=true}


\begin{document}

\renewcommand{\O}{\mathcal{O}}


\biaoti{\kaishu \textbf{上海应用技术大学 2023--2024 学年第二学期}}
\kemu{\textbf{《{\kaishu 强化学习}》期末试卷 \quad (B) 参考答案}}
\setxz{40}{20}{2}
\setpd{20}{10}{2}
\setjd{40}{4}
\xinxi{100}{90}

\maketitle

\begin{questions}

\xuanze

\question 强化学习的目标是:
\begin{choices}
    \choice 最小化损失函数
    \choice 最大化准确率
    \CorrectChoice 最大化累积奖励
    \choice 最小化状态数量
\end{choices}

\question 在马尔可夫决策过程(MDP)中,状态转移概率函数P(s'|s,a)表示:
\begin{choices}
    \choice 在状态s执行动作a获得的奖励
    \CorrectChoice 从状态s执行动作a后到达状态s'的概率
    \choice 状态s的价值
    \choice 动作a在状态s下的价值
\end{choices}

\question 在强化学习中,最优动作值函数Q*(s,a)的定义是:
\begin{choices}
    \choice 在状态s下执行动作a后获得的即时奖励
    \CorrectChoice 在状态s下执行动作a后的最大期望回报
    \choice 状态s的最大期望回报
    \choice 遵循任意策略时在状态s执行动作a的期望回报
\end{choices}

\question 以下哪种算法属于无模型(Model-free)强化学习方法?
\begin{choices}
    \choice 策略迭代
    \choice 值迭代
    \CorrectChoice Q-Learning
    \choice 动态规划
\end{choices}

\question 强化学习中的"探索与利用"权衡是指:
\begin{choices}
    \choice 在训练与测试间的平衡
    \choice 在奖励与惩罚间的平衡
    \CorrectChoice 在尝试新动作与选择已知最优动作间的平衡
    \choice 在在线学习与离线学习间的平衡
\end{choices}

\question 在时序差分(TD)学习中,与蒙特卡洛方法相比的主要优势是:
\begin{choices}
    \choice 更低的偏差
    \CorrectChoice 能够从不完整的回合中学习
    \choice 只使用实际奖励不使用估计值
    \choice 不需要探索策略
\end{choices}

\question 在深度强化学习中,目标网络(Target Network)的主要作用是:
\begin{choices}
    \choice 加速梯度下降
    \choice 提高探索效率
    \choice 增加神经网络深度
    \CorrectChoice 稳定训练过程
\end{choices}

\question 策略梯度方法与基于值的方法相比的主要优势是:
\begin{choices}
    \CorrectChoice 能够直接处理连续动作空间
    \choice 计算效率更高
    \choice 需要更少的训练样本
    \choice 不需要探索策略
\end{choices}

\question Actor-Critic方法的核心思想是:
\begin{choices}
    \choice 结合蒙特卡洛和时序差分学习
    \CorrectChoice 结合策略梯度和值函数逼近
    \choice 结合在线学习和离线学习
    \choice 结合监督学习和无监督学习
\end{choices}

\question 在强化学习中,什么是回报(Return)?
\begin{choices}
    \choice 即时奖励
    \choice 所有状态的平均奖励
    \CorrectChoice 从当前时刻开始的折扣累积奖励
    \choice 环境的状态转移函数
\end{choices}

\question 强化学习中的同策略(On-policy)学习和异策略(Off-policy)学习的主要区别是:
\begin{choices}
    \choice 同策略使用神经网络,异策略使用表格
    \choice 同策略需要环境模型,异策略不需要
    \CorrectChoice 同策略使用当前正在学习的策略采集数据,异策略可以使用不同策略采集数据
    \choice 同策略适用于离散状态,异策略适用于连续状态
\end{choices}

\question 以下哪种算法是异策略(Off-policy)强化学习方法?
\begin{choices}
    \choice SARSA
    \choice Actor-Critic
    \choice 蒙特卡洛策略评估
    \CorrectChoice Q-Learning
\end{choices}

\question 深度Q网络(DQN)中的经验回放(Experience Replay)机制主要解决的问题是:
\begin{choices}
    \choice 存储容量限制
    \CorrectChoice 数据样本之间的相关性
    \choice 神经网络训练速度慢
    \choice 环境随机性大
\end{choices}

\question 在蒙特卡洛方法中,对状态价值的更新是基于:
\begin{choices}
    \choice 后续一步的奖励与下一状态的估计值
    \CorrectChoice 后续所有时间步的实际奖励
    \choice 仅当前状态的即时奖励
    \choice 环境模型预测的奖励
\end{choices}

\question 在策略梯度方法中,优势函数(Advantage Function)的作用是:
\begin{choices}
    \choice 加速收敛
    \CorrectChoice 减小方差
    \choice 增加探索
    \choice 降低训练复杂度
\end{choices}

\question 时序差分(TD)学习中的"自举(Bootstrapping)"是指:
\begin{choices}
    \choice 使用多个不同的学习率
    \CorrectChoice 使用当前估计更新当前估计
    \choice 使用多个网络参数副本
    \choice 在多个环境同时训练
\end{choices}

\question 在强化学习中,前交互学习(Off-line Learning)指的是:
\begin{choices}
    \choice 在真实环境中学习
    \choice 在线上进行实时学习
    \CorrectChoice 使用事先收集的数据集进行学习
    \choice 不更新模型参数的学习
\end{choices}

\question SARSA算法名称的由来是:
\begin{choices}
    \choice 算法发明者的姓名缩写
    \choice 随机搜索与行动算法的缩写
    \choice 算法更新使用的数据元组(S,A,R,S',A')
    \CorrectChoice 状态动作奖励状态动作的英文缩写
\end{choices}

\question 以下哪种方法不适合解决连续动作空间的问题?
\begin{choices}
    \choice 策略梯度
    \choice Actor-Critic
    \CorrectChoice 原始形式的DQN
    \choice DDPG
\end{choices}

\question 强化学习中的"信用分配问题"(Credit Assignment Problem)是指:
\begin{choices}
    \choice 如何确定每个状态的奖励值
    \choice 如何分配计算资源
    \CorrectChoice 如何确定哪些动作导致了最终的结果
    \choice 如何为智能体设计奖励函数
\end{choices}

\panduan

\question 强化学习中的奖励值一定是正数,负奖励表示惩罚。
\answerNo{强化学习中的奖励可以是正数、负数或零。奖励只是一个信号,表示动作的好坏,不一定要分为"奖励"和"惩罚"两类。}

\question 策略迭代(Policy Iteration)算法通常比值迭代(Value Iteration)算法需要更少的迭代次数就能收敛。
\answerYes{策略迭代通常需要更少的迭代次数,因为每次迭代都进行完整的策略评估,但每次迭代的计算成本更高。}

\question 在强化学习中,折扣因子γ越大,表示智能体越重视未来奖励。
\answerYes{折扣因子γ控制未来奖励的权重,γ越大(接近1),表示智能体越重视长期回报;γ越小(接近0),表示越注重短期回报。}

\question 时序差分(TD)学习方法同时具有蒙特卡洛方法和动态规划方法的优点。
\answerYes{TD学习结合了MC方法的实际采样和DP方法的自举估计特点,不需要环境模型(像MC),也不需要等到回合结束就可以更新(像DP)。}

\question DQN算法主要用于解决连续动作空间问题。
\answerNo{原始DQN主要用于离散动作空间问题。处理连续动作空间通常需要策略梯度方法或其变体,如DDPG。}

\question 在强化学习中,价值函数(Value Function)用于评估状态或状态-动作对的好坏。
\answerYes{价值函数评估状态或状态-动作对的长期价值,通常表示为从该状态开始,遵循特定策略所能获得的期望累积奖励。}

\question Actor-Critic方法中,Actor负责选择动作,Critic负责评估动作。
\answerYes{Actor负责策略函数,决定在各状态下采取什么动作;Critic负责评估这些状态或动作的价值,为Actor提供反馈。}

\question 异策略(Off-policy)学习方法比同策略(On-policy)学习方法通常有更低的样本效率。
\answerNo{异策略方法通常具有更高的样本效率,因为它们可以重用从不同策略收集的数据,而同策略方法只能使用当前策略生成的数据。}

\question 强化学习算法的收敛性严重依赖于探索策略的选择。
\answerYes{探索策略对算法收敛性有重要影响,不足的探索可能导致局部最优,过度探索则可能影响收敛速度。}

\question 蒙特卡洛方法不需要环境的马尔可夫性质,而时序差分方法依赖于此。
\answerYes{蒙特卡洛方法只依赖于回合的总回报,不关心中间状态转移,因此不需要马尔可夫性质;时序差分方法使用下一状态的估计值,依赖于马尔可夫性质。}

\jianda

\question[10] 探索与利用(Exploration & Exploitation):
\begin{solution}
\begin{enumerate}
    \item 什么是强化学习中的"探索与利用"问题?为什么它在强化学习中很重要?(3分)
    
    探索与利用问题是指智能体在决策过程中面临的两难选择:是选择已知的最优动作以获取即时高回报(利用),还是尝试未充分了解的动作以获取新信息(探索)。
    
    重要性:
    - 只利用会导致策略陷入局部最优,错过潜在更好的策略
    - 只探索会导致大量低效决策,无法充分利用已学到的知识
    - 适当平衡是找到全局最优策略的关键
    - 在奖励稀疏或延迟的环境中尤为重要,因为需要充分探索才能发现有效路径
    
    \item 描述两种常用的探索策略,并分析它们的优缺点(4分)
    
    1. ε-贪心策略:
       - 原理:以1-ε的概率选择估计值最高的动作(利用),以ε的概率随机选择动作(探索)
       - 优点:简单易实现,直观易理解,有理论收敛保证
       - 缺点:探索是完全随机的,不考虑动作的不确定性或潜在价值;对所有非贪心动作给予相同概率
       - 适用场景:动作空间较小的问题
    
    2. 玻尔兹曼(Boltzmann/Softmax)探索:
       - 原理:根据动作的估计值按概率分布选择动作,估计值高的动作被选择概率更大
       - 优点:探索更有针对性,考虑动作间的价值差异;通过温度参数可控制探索-利用平衡
       - 缺点:计算复杂度高,温度参数需要精心调整
       - 适用场景:需要更精细探索控制的问题
    
    其他可能的回答(UCB、Thompson采样等)也可酌情给分。
    
    \item 如何在训练过程中平衡探索与利用?(3分)
    
    1. 动态调整探索参数:
       - 在训练初期使用高探索率,随着学习进展逐渐降低(如ε衰减)
       - 基于学习进度或性能指标自适应调整探索参数
    
    2. 采用更先进的探索策略:
       - 基于不确定性的探索(如UCB算法)
       - 基于状态新颖性的探索(如内在动机)
       - 参数空间噪声(如参数噪声探索)
    
    3. 专门的探索阶段设计:
       - 预训练探索阶段收集多样化经验
       - 分层探索策略(宏观探索与微观利用结合)
       - 探索奖励与任务奖励分离
    
    一个好的平衡策略应根据具体任务特点、环境复杂度和训练阶段调整,没有放之四海而皆准的方法。
\end{enumerate}
\end{solution}

\question[10] 策略梯度方法(Policy Gradient Methods):
\begin{solution}
\begin{enumerate}
    \item 简述策略梯度方法的基本思想(3分)
    
    策略梯度方法的基本思想是直接对策略函数进行参数化,并通过梯度上升方法优化策略参数,以最大化预期回报。
    
    核心公式是策略梯度定理:$\nabla_\theta J(\theta) = \mathbb{E}_{\pi_\theta} [\nabla_\theta \log \pi_\theta(a|s) \cdot R]$
    
    其中:
    - $\pi_\theta(a|s)$是参数为$\theta$的策略函数,表示在状态$s$选择动作$a$的概率
    - $J(\theta)$是目标函数,表示策略$\pi_\theta$的期望回报
    - $R$是回报(可以是总回报、折扣回报或优势函数)
    
    策略梯度方法的直观理解:增加导致高回报的动作被选择的概率,减少导致低回报的动作被选择的概率。
    
    \item 策略梯度方法与基于值的方法(如Q-Learning)相比有哪些优势和劣势?(4分)
    
    优势:
    - 能够直接学习随机策略,对部分可观察环境、对抗环境更有效
    - 可以自然处理连续动作空间,不需要离散化或求解复杂的最大化问题
    - 策略变化更平滑,不会因为值函数微小变化导致策略剧烈变化
    - 在高维或连续动作空间中更有效率,因为不需要评估所有可能动作
    
    劣势:
    - 通常收敛到局部最优而非全局最优
    - 梯度估计通常具有高方差,需要更多样本才能可靠学习
    - 样本效率通常低于基于值的方法,需要更多交互数据
    - 对超参数选择和初始化更敏感,调参难度大
    
    \item 为什么在策略梯度方法中通常需要使用基线(Baseline)技术?(3分)
    
    基线技术的使用原因:
    
    1. 减小梯度估计的方差:
       - 策略梯度估计通常具有高方差,导致训练不稳定
       - 基线减少了回报的绝对规模,只关注相对优势,显著降低梯度估计的方差
    
    2. 保持无偏估计:
       - 理论上证明,添加不依赖于动作的基线不会引入偏差
       - 基线的引入不改变梯度的期望值,只减小其方差
    
    3. 增强学习信号:
       - 通过对比动作的相对优劣,而非绝对回报,提供更明确的学习信号
       - 帮助区分在同一状态下不同动作的相对价值
    
    常用的基线选择是状态值函数$V(s)$,此时策略梯度更新使用的是优势函数$A(s,a) = Q(s,a) - V(s)$,衡量特定动作相对于平均表现的好坏。
\end{enumerate}
\end{solution}

\question[10] Actor-Critic方法:
\begin{solution}
\begin{enumerate}
    \item 描述Actor-Critic方法的基本框架和工作原理(4分)
    
    Actor-Critic方法是结合策略梯度和值函数逼近的混合架构,由两个主要组件组成:
    
    1. 基本框架:
       - Actor:参数化策略函数$\pi_\theta(a|s)$,负责选择动作
       - Critic:参数化值函数$V_w(s)$或$Q_w(s,a)$,负责评估状态或动作的价值
    
    2. 工作原理:
       - Actor根据当前策略在环境中执行动作
       - Critic观察结果,评估Actor的表现,计算TD误差
       - Actor使用Critic的评估(通常是TD误差)作为学习信号来改进策略
       - Critic也不断更新其价值估计,以提供更准确的评估
    
    3. 更新过程:
       - Critic更新:最小化TD误差的平方,$L(w) = (r + \gamma V_w(s') - V_w(s))^2$
       - Actor更新:使用TD误差作为优势估计,$\theta \leftarrow \theta + \alpha \delta \nabla_\theta \log \pi_\theta(a|s)$
       - 其中$\delta = r + \gamma V_w(s') - V_w(s)$是TD误差
    
    \item Actor和Critic各自的作用是什么?它们如何协同工作?(3分)
    
    Actor的作用:
    - 负责策略学习,决定在每个状态下应采取何种动作
    - 根据当前策略与环境交互,生成经验数据
    - 根据Critic的反馈调整策略参数,提高动作选择的质量
    
    Critic的作用:
    - 负责值函数学习,评估状态或状态-动作对的价值
    - 计算TD误差或优势估计,为Actor提供更新信号
    - 减少策略梯度估计的方差,提高学习稳定性
    
    协同工作方式:
    - 形成反馈循环:Actor执行,Critic评估,Actor根据评估改进
    - 相互促进:Actor生成数据帮助Critic学习更准确的值函数,Critic提供更精确的学习信号帮助Actor改进策略
    - 分工明确:Actor专注于"如何行动",Critic专注于"结果如何"
    
    \item Actor-Critic方法如何结合策略梯度和时序差分学习的优点?(3分)
    
    结合策略梯度的优点:
    - 可以直接处理连续动作空间
    - 能够学习随机策略,适应部分可观察环境
    - 策略更新平滑,避免小的值函数变化导致的策略剧烈变化
    
    结合时序差分学习的优点:
    - 无需等待回合结束就可以进行在线更新
    - 通过自举提高样本效率,比纯蒙特卡洛方法更数据高效
    - TD误差提供即时反馈,加速学习过程
    
    具体结合方式:
    - 使用TD误差作为优势函数的估计,替代传统策略梯度中的回报
    - 将Critic学到的值函数作为基线,减小策略梯度的方差
    - 同时更新两个网络,使它们协同改进
    
    这种结合弥补了单独使用策略梯度或值函数方法的不足,提高了算法的稳定性、样本效率和性能。
\end{enumerate}
\end{solution}

\question[10] 强化学习应用场景分析:
\begin{solution}
\begin{enumerate}
    \item 强化学习在游戏、机器人控制和资源管理等领域的应用各有什么特点?(4分)
    
    游戏领域应用特点:
    - 有明确的规则和奖励信号(分数、胜负)
    - 可以快速模拟大量数据,训练环境稳定
    - 无安全风险,允许充分探索
    - 典型应用:AlphaGo(围棋)、OpenAI Five(DOTA2)、Atari游戏
    - 常用算法:DQN及其变体、AlphaZero算法
    
    机器人控制领域特点:
    - 需要处理连续状态和动作空间
    - 物理世界交互带来安全挑战和样本获取成本高
    - 需要平衡探索与安全性
    - 典型应用:机械臂操作、步行机器人、无人机控制
    - 常用算法:DDPG、PPO、SAC等适合连续控制的算法
    
    资源管理领域特点:
    - 优化目标通常是长期效益与短期成本的平衡
    - 决策影响范围广,错误成本高
    - 通常有复杂约束条件
    - 典型应用:数据中心能源管理、通信网络优化、供应链管理
    - 常用算法:基于约束的强化学习、分层强化学习
    
    \item 在面对不同类型的强化学习问题时,如何选择合适的算法?请给出方法和依据(3分)
    
    选择算法的主要考虑因素和方法:
    
    1. 状态与动作空间特性:
       - 离散、有限动作空间:可使用Q-learning、DQN等基于值的方法
       - 连续动作空间:优先考虑策略梯度法(如PPO、DDPG、SAC)
       - 高维状态空间:需要函数逼近(如DQN)或特征提取(如CNN+RL)
    
    2. 样本效率需求:
       - 实际物理系统(样本昂贵):模型型方法或高效异策略方法(如SAC)
       - 仿真环境(样本廉价):可以使用样本效率较低但稳定的算法(如PPO)
    
    3. 探索难度:
       - 稀疏奖励问题:需要高效探索策略的算法(如好奇心驱动方法)
       - 陷阱多的环境:需要保守策略更新的算法(如TRPO、PPO)
    
    4. 任务特性:
       - 需要随机策略:策略梯度方法
       - 安全敏感:基于约束的RL或保守策略更新的方法
       - 多任务学习:分层强化学习或元学习方法
    
    5. 实用建议:
       - 从简单算法开始,逐步增加复杂性
       - 考虑算法的实现难度、计算需求和调参复杂度
       - 参考相似问题的成功案例
    
    \item 在实际应用中,强化学习面临哪些常见挑战?如何应对这些挑战?(3分)
    
    常见挑战与应对策略:
    
    1. 样本效率低:
       - 挑战:需要大量交互数据才能学习有效策略
       - 应对:使用模型型方法、经验回放、预训练、迁移学习,或结合模仿学习
    
    2. 泛化能力弱:
       - 挑战:学到的策略难以应对训练环境外的情况
       - 应对:领域随机化、对抗训练、元学习、增强状态表示的多样性
    
    3. 安全与约束:
       - 挑战:探索过程可能导致危险行为
       - 应对:基于约束的强化学习、安全层、人类监督、模拟环境预训练
    
    4. 稳定性与收敛性:
       - 挑战:训练过程不稳定,结果难以复现
       - 应对:保守策略更新(PPO/TRPO)、集成学习、多次运行取平均
    
    5. 奖励设计困难:
       - 挑战:设计能引导期望行为的奖励函数很复杂
       - 应对:逆强化学习、多目标强化学习、层次化奖励
    
    6. 现实世界与仿真差距:
       - 挑战:仿真训练的策略在实际环境表现不佳
       - 应对:领域自适应、渐进式部署、混合实际与仿真数据
\end{enumerate}
\end{solution}

\end{questions}

\end{document} 