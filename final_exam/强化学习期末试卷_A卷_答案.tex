\documentclass[zihao=5,answers]{BHCexam}
\usepackage{ctex}
\usepackage{multirow}

%\linespread{1.2}

% \usepackage{minted}
% \setminted{mathescape, style = black, linenos=true}


\begin{document}

\renewcommand{\O}{\mathcal{O}}

\biaoti{\kaishu \textbf{上海应用技术大学 2023--2024 学年第二学期}}
\kemu{\textbf{《{\kaishu 强化学习}》期末试卷 \quad (A) 参考答案}}
\setxz{40}{20}{2}
\setpd{20}{10}{2}
\setjd{40}{4}
\xinxi{100}{90}

\maketitle

%%%%%%%%%%%%%%%%%%%%%%%%%%%%%%%%%%%%%%%%%%%%%%

\begin{questions}

\xuanze

\question 在强化学习中,智能体与环境交互的基本元素不包括:
\begin{choices}
    \choice 状态
    \choice 动作
    \choice 奖励
    \CorrectChoice 损失函数
\end{choices}

\question 马尔可夫决策过程(MDP)中,折扣因子γ的作用是:
\begin{choices}
    \choice 控制状态转移概率
    \CorrectChoice 平衡当前与未来奖励的重要性
    \choice 定义状态空间大小
    \choice 计算累积奖励的绝对值
\end{choices}

\question 在强化学习中,环境模型是指:
\begin{choices}
    \choice 智能体的学习策略
    \choice 价值函数的近似方法
    \CorrectChoice 状态转移概率和奖励函数
    \choice 探索与利用的平衡方法
\end{choices}

\question 值函数迭代算法属于:
\begin{choices}
    \CorrectChoice 基于模型的方法
    \choice 无模型方法
    \choice 策略梯度方法
    \choice 监督学习方法
\end{choices}

\question Q-learning是一种:
\begin{choices}
    \choice 基于模型的方法
    \choice 策略优化方法
    \CorrectChoice 时序差分学习
    \choice 蒙特卡洛方法
\end{choices}

\question ε-贪心策略的主要目的是:
\begin{choices}
    \choice 加速值函数收敛
    \CorrectChoice 平衡探索与利用
    \choice 减少计算复杂度
    \choice 降低神经网络参数数量
\end{choices}

\question 在强化学习中,"回合"(Episode)的定义是:
\begin{choices}
    \choice 单次状态转移
    \choice 累积奖励计算周期
    \CorrectChoice 从初始状态到终止状态的完整序列
    \choice 值函数更新的一个批次
\end{choices}

\question SARSA与Q-learning的主要区别在于:
\begin{choices}
    \choice SARSA使用神经网络,Q-learning使用表格
    \choice SARSA用于连续状态空间,Q-learning用于离散状态空间
    \CorrectChoice SARSA是同策略(on-policy)方法,Q-learning是异策略(off-policy)方法
    \choice SARSA需要环境模型,Q-learning不需要
\end{choices}

\question 下列哪种方法可以直接处理连续动作空间?
\begin{choices}
    \choice 基本Q-learning
    \choice 表格式SARSA
    \CorrectChoice 策略梯度法
    \choice 蒙特卡洛树搜索
\end{choices}

\question 深度Q网络(DQN)相比传统Q-learning的主要创新在于:
\begin{choices}
    \choice 使用折扣因子
    \CorrectChoice 使用神经网络逼近Q函数,并引入经验回放和目标网络
    \choice 采用模型预测
    \choice 同时学习多个任务
\end{choices}

\question 下列关于强化学习的描述,错误的是:
\begin{choices}
    \choice 强化学习中智能体通过与环境交互来学习
    \choice 强化学习可以处理序列决策问题
    \CorrectChoice 强化学习总是需要环境转移模型
    \choice 强化学习的目标是最大化累积奖励
\end{choices}

\question 强化学习中,动作值函数Q(s,a)表示:
\begin{choices}
    \choice 在状态s下执行动作a的即时奖励
    \CorrectChoice 在状态s下执行动作a后,遵循当前策略的期望累积奖励
    \choice 状态s的价值
    \choice 执行动作a的概率
\end{choices}

\question 时序差分(TD)学习与蒙特卡洛方法相比:
\begin{choices}
    \choice 总是有更高的精度
    \choice 总是需要更多的样本
    \CorrectChoice 不需要等到回合结束就可以更新估计
    \choice 不依赖于马尔可夫性质
\end{choices}

\question "经验回放"(Experience Replay)在DQN中的作用是:
\begin{choices}
    \choice 提高探索效率
    \CorrectChoice 打破样本之间的相关性,提高学习效率
    \choice 减少神经网络参数数量
    \choice 避免环境随机性
\end{choices}

\question Actor-Critic方法中:
\begin{choices}
    \choice Actor负责评估动作,Critic负责选择动作
    \CorrectChoice Actor负责选择动作,Critic负责评估状态或动作
    \choice Actor估计状态值函数,Critic估计动作值函数
    \choice Actor和Critic使用相同的网络结构
\end{choices}

\question 深度确定性策略梯度(DDPG)算法主要用于解决:
\begin{choices}
    \choice 多智能体协作问题
    \choice 部分可观察问题
    \CorrectChoice 连续动作空间问题
    \choice 稀疏奖励问题
\end{choices}

\question 在强化学习中,"自举"(Bootstrapping)是指:
\begin{choices}
    \CorrectChoice 使用估计值来更新估计值
    \choice 从多个初始状态同时学习
    \choice 使用多个智能体并行学习
    \choice 重复采样训练数据
\end{choices}

\question 在策略梯度方法中,基线(Baseline)的引入主要是为了:
\begin{choices}
    \choice 加速训练
    \choice 提高探索效率
    \CorrectChoice 减小梯度估计的方差
    \choice 解决非平稳环境问题
\end{choices}

\question 强化学习中的"信用分配问题"(Credit Assignment Problem)是指:
\begin{choices}
    \choice 如何为不同的智能体分配任务
    \CorrectChoice 如何确定哪些动作对最终结果贡献最大
    \choice 如何分配计算资源
    \choice 如何设计奖励函数
\end{choices}

\question 在强化学习中,策略函数π(a|s)表示:
\begin{choices}
    \choice 状态s的价值
    \choice 动作a的价值
    \CorrectChoice 在状态s下选择动作a的概率
    \choice 从状态s转移到下一状态的概率
\end{choices}

\panduan

\question 强化学习中,智能体必须知道环境的转移模型才能学习策略。
\answerNo{强化学习分为基于模型和无模型方法,无模型方法(如Q-learning、SARSA)不需要知道环境转移模型也能学习策略。}

\question 折扣因子γ=0意味着智能体只关注即时奖励,不考虑未来奖励。
\answerYes{γ=0时,未来奖励的折扣值为0,智能体只考虑当前步骤的即时奖励。}

\question Q-learning算法是一种异策略(off-policy)强化学习方法。
\answerYes{Q-learning使用贪心策略选择最优动作进行更新,而使用另一种策略(如ε-贪心)进行探索,因此是异策略方法。}

\question 在蒙特卡洛方法中,必须等到回合结束才能更新价值函数。
\answerYes{蒙特卡洛方法基于完整回合的实际回报来更新价值函数,所以必须等到回合结束才能进行更新。}

\question 在策略梯度方法中,直接对策略函数进行参数化和优化。
\answerYes{策略梯度方法直接参数化策略函数,并通过梯度上升方法优化策略参数以最大化期望回报。}

\question 深度Q网络(DQN)中的目标网络(Target Network)与主网络使用相同的参数。
\answerNo{目标网络参数是主网络参数的延迟更新版本,两者不同步更新,目的是稳定训练过程。}

\question SARSA算法计算的是当前策略下的最优动作值函数。
\answerNo{SARSA估计的是当前策略(通常是ε-贪心策略)下的动作值函数,不一定是最优动作值函数。}

\question Actor-Critic方法结合了策略梯度和时序差分学习的优点。
\answerYes{Actor使用策略梯度方法学习策略,Critic使用时序差分方法学习值函数,结合了两种方法的优点。}

\question 在所有强化学习任务中,状态空间和动作空间必须是离散的。
\answerNo{强化学习可以处理连续的状态空间和动作空间,例如通过函数逼近方法或直接优化连续策略。}

\question 基于值的方法比策略梯度方法更适合处理连续动作空间的问题。
\answerNo{基于值的方法在连续动作空间中面临最大化问题,而策略梯度方法可以直接输出连续动作,更适合处理连续动作空间。}

\jianda

\question[10] 基础强化学习算法比较:Q-learning与SARSA
\begin{solution}
\begin{enumerate}
    \item 基本原理比较(3分):
    
    Q-learning和SARSA都是时序差分学习方法,用于估计动作值函数。
    
    Q-learning更新公式:$Q(s_t,a_t) \leftarrow Q(s_t,a_t) + \alpha [r_{t+1} + \gamma \max_a Q(s_{t+1},a) - Q(s_t,a_t)]$
    
    SARSA更新公式:$Q(s_t,a_t) \leftarrow Q(s_t,a_t) + \alpha [r_{t+1} + \gamma Q(s_{t+1},a_{t+1}) - Q(s_t,a_t)]$
    
    主要区别在于Q-learning使用下一状态的最大Q值($\max_a Q(s_{t+1},a)$),而SARSA使用下一状态实际选择的动作的Q值($Q(s_{t+1},a_{t+1})$)。
    
    \item 同策略vs异策略(3分):
    
    Q-learning是异策略(off-policy)方法:评估的是最优策略(贪心策略),但使用另一种策略(通常是ε-贪心)进行探索和数据收集。
    
    SARSA是同策略(on-policy)方法:评估的策略与用于探索的策略相同,都是当前的行为策略(通常是ε-贪心)。
    
    \item 性能特点比较(4分):
    
    Q-learning:
    - 优点:可以直接学习最优策略,不受探索策略限制
    - 缺点:在某些安全关键或风险规避场景可能过于乐观
    - 适用场景:可以承受探索风险的环境
    
    SARSA:
    - 优点:考虑探索策略的风险,更为保守
    - 缺点:收敛到的策略受探索策略影响
    - 适用场景:需要在学习过程中避免高风险状态的任务
    
    实例说明:在悬崖行走问题中,Q-learning倾向于沿悬崖边缘行走(理论最优但风险高),SARSA则会选择更远离悬崖的安全路径。
\end{enumerate}
\end{solution}

\question[10] 深度强化学习:DQN的关键创新点及其重要性
\begin{solution}
\begin{enumerate}
    \item 神经网络函数逼近(2分):
    
    DQN使用深度神经网络逼近Q函数,解决了传统表格方法无法处理高维状态空间的问题。神经网络能够自动提取特征并泛化到未见过的状态,使算法可以应用于复杂环境,如图像输入的Atari游戏。
    
    \item 经验回放(Experience Replay)(3分):
    
    DQN将智能体的经验(状态、动作、奖励、下一状态)存储在回放缓冲区中,训练时随机采样批次进行学习。
    
    重要性:
    - 打破样本间的时序相关性,提高训练稳定性
    - 提高样本利用效率,每个经验可被多次使用
    - 减轻非平稳分布问题,使神经网络训练更稳定
    
    \item 目标网络(Target Network)(3分):
    
    DQN使用两个网络:主网络用于选择动作和更新,目标网络用于计算TD目标。目标网络参数定期从主网络复制,而不是每步更新。
    
    重要性:
    - 减缓目标值的变化,增加训练稳定性
    - 避免震荡和发散问题
    - 减轻过度估计(overestimation)问题
    
    \item 其他改进及影响(2分):
    
    - 预处理和框架堆叠:减少输入维度并提供时序信息
    - ε-贪心策略:平衡探索与利用
    
    DQN的影响:
    - 首次展示了深度强化学习在复杂环境中的有效性
    - 为后续Double DQN、Dueling DQN、Rainbow等算法奠定基础
    - 推动了深度强化学习在游戏、机器人控制等领域的应用
\end{enumerate}
\end{solution}

\question[10] 策略梯度方法(Policy Gradient):
\begin{solution}
\begin{enumerate}
    \item 基本原理(3分):
    
    策略梯度方法直接参数化策略函数$\pi_\theta(a|s)$,并通过梯度上升方法优化参数$\theta$以最大化期望回报$J(\theta)$。
    
    策略梯度定理给出了目标函数的梯度:
    $\nabla_\theta J(\theta) = \mathbb{E}_{\pi_\theta} [\nabla_\theta \log \pi_\theta(a|s) \cdot Q^{\pi_\theta}(s,a)]$
    
    直观理解:增加导致高回报动作的概率,减少导致低回报动作的概率。
    
    \item 与基于值的方法比较(4分):
    
    优势:
    - 可以直接处理连续动作空间,不需要离散化或最大化操作
    - 可以学习随机策略,有助于探索和应对部分可观察环境
    - 策略更新更为稳定,不会因小的值函数变化导致策略剧烈变化
    
    劣势:
    - 通常收敛到局部最优而非全局最优
    - 样本效率通常低于基于值的方法
    - 梯度估计方差较大,需要大量样本
    
    \item 基线(Baseline)技术(3分):
    
    策略梯度可以重写为:$\nabla_\theta J(\theta) = \mathbb{E}_{\pi_\theta} [\nabla_\theta \log \pi_\theta(a|s) \cdot (Q^{\pi_\theta}(s,a) - b(s))]$
    
    其中$b(s)$是状态相关的基线函数,通常选择状态值函数$V^{\pi_\theta}(s)$。
    
    引入基线的原因:
    - 减小梯度估计的方差,使训练更稳定
    - 不改变梯度的期望值(无偏估计)
    - 突出动作的相对优势,而非绝对回报
    
    当使用状态值函数作为基线时,$(Q^{\pi_\theta}(s,a) - V^{\pi_\theta}(s))$实际上是优势函数(Advantage),衡量特定动作相对于平均表现的好坏。
\end{enumerate}
\end{solution}

\question[10] Actor-Critic架构与算法:
\begin{solution}
\begin{enumerate}
    \item 基本框架与组成(3分):
    
    Actor-Critic结合了策略梯度和时序差分学习,由两个主要组件组成:
    
    - Actor:负责学习策略函数$\pi_\theta(a|s)$,决定在每个状态应采取何种动作
    - Critic:负责学习价值函数(通常是状态值函数$V_w(s)$或动作值函数$Q_w(s,a)$),评估当前策略的好坏
    
    两部分可以使用不同的神经网络或同一网络的不同部分。
    
    \item 工作原理与更新过程(4分):
    
    基本工作流程:
    1. Actor根据当前策略$\pi_\theta(a|s)$在状态$s$选择动作$a$
    2. 执行动作$a$,观察奖励$r$和下一状态$s'$
    3. Critic计算TD误差:$\delta = r + \gamma V_w(s') - V_w(s)$
    4. 使用TD误差更新Critic:$w \leftarrow w + \alpha_w \cdot \delta \cdot \nabla_w V_w(s)$
    5. 使用Critic的评估更新Actor:$\theta \leftarrow \theta + \alpha_\theta \cdot \delta \cdot \nabla_\theta \log \pi_\theta(a|s)$
    
    关键特点:
    - TD误差$\delta$作为优势估计,指导策略更新
    - Actor和Critic相互协作,形成反馈循环
    
    \item 算法变体与应用(3分):
    
    主要变体:
    - A2C/A3C (Advantage Actor-Critic):使用多步回报估计优势
    - DDPG (Deep Deterministic Policy Gradient):适用于连续动作空间的确定性策略梯度方法
    - PPO (Proximal Policy Optimization):通过约束策略更新幅度提高稳定性
    - SAC (Soft Actor-Critic):增加策略熵以鼓励探索
    
    应用场景:
    - 机器人控制:需要平滑、连续动作的任务
    - 游戏AI:需要快速决策和在线学习
    - 资源分配:需要在复杂环境中做出序列决策
\end{enumerate}
\end{solution}

\end{questions}

\end{document} 